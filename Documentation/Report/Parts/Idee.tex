\authoredSection{michael}{App-Idee}

\subsection{Problematik}
	Mit Rücksicht auf oben genannte Aspekte haben wir uns in der Ideenfindungsphase auf Lokalität als entscheidendes Schlagwort konzentriert. Apps wie etwa Yelp\footnote{\url{http://www.yelp.de/}, Abruf am 23.02.2014 um 16:15 Uhr, siehe auch Abbildung \ref{fig:Yelp}}, welche Bezug auf lokale Unternehmen nehmen und Benutzern die Möglichkeit geben, sich darüber auszutauschen, sind beispielhaft für eine erfolgreiche Umsetzung dieses Konzepts.
	
	Für das Bankgeschäft ergab sich jedoch das Problem, dass die eigenen finanziellen Verhältnisse in den innersten persönlichen Lebensbereich fallen und in der Regel darüber zumindest weniger Austausch stattfindet als über Restaurants, Nachtclubs oder Freizeitaktivitäten. 

\begin{figure}[h]
	\centering
	\includegraphics[scale=0.2]{Pictures/Yelp}
	\caption{Screenshot der Yelp-App \label{fig:Yelp}}
\end{figure}

	Erste Überlegungen zielten in Richtung interaktiver Beratungsdienstleistungen, welche jederzeit über Videotelefonie Kontakt zum individuell vertrauten Berater einer bestimmten Filiale herstellen sollten. Der individuelle Berater ist aber kein Alleinstellungsmerkmal der Filialbank mehr, eine virtuelle Filiale ist ohne weiteres denkbar.

	Allerdings, und hier haben wir die Chance unserer Anwendung gesehen, bedeutet die technische Möglichkeit eines Vertriebsweges nicht gleich, dass dieser auch angenommen wird. Eine von uns im Sinne einer Markterkundung durchgeführte Umfrage ergab ein gewisses Misstrauen gegenüber der Idee, Finanztransaktionen per App durchzuführen oder überhaupt eine Bank-App zu benutzen.
	
	Gründe hierfür könnten vielfältig sein; beispielsweise könnten die befragten Studenten am Informatikcampus eine besondere Sensibilität bezüglich Aspekten des Datenschutzes haben. Eine von der Unternehmensberatung „Bain and Company“ durchgeführte Studie zur Frage, wie digitalisierte Vertriebswege das Privatkundengeschäft der Banken zukünftig prägen könnten, ergab jedoch einen ähnlichen Tenor\footnote{\url{http://www.bain.de/Images/Retail\_Banking\_II\_Digitalisierung\_ES.pdf}, Abruf am 23.02.2014 um 16:30 Uhr.}. Insbesondere ergab die Studie, dass für Geldanlagen dem persönlichen Gespräch der Vorzug vor digitaler Beratung gegeben würde. Somit könnte es trotz aller technischen Möglichkeiten auch langfristig noch einen Markt für persönliche Beratung von Angesicht zu Angesicht geben.
 
\subsection{Vision}
	Schließlich entstand die Vision einer Anwendung, die nicht versucht, aus einer Filialbank eine Direktbank mit Filialen als Zusatzangebot zu machen, sondern die die bestehenden Produkte des Filialgeschäfts dem Kunden näher bringt und ihm Grund gibt, in eine Filiale zu gehen. 
	
	Als Kernprodukte und Kompetenzen haben wir das klassische Sparbuch, die Anlageberatung sowie die Finanzierung ausgemacht. Das Sparbuch bietet scheinbar keine Funktionalität, die einen konkreten Vorteil gegenüber beispielsweise dem Tagesgeldkonto einer Direktbank hätte. Wie aber einführend erläutert, ist dies nicht ausschlaggebend. Denn ungeachtet seiner Nachteile ist das Sparbuch populär; vielleicht auch, weil in den Verwerfungen der europäischen Finanzkrise hier besondere Sicherheit vermutet wird. Diesem Sicherheitsbedürfnis der Kunden wollten wir dabei konsequent nicht nur Architektur, sondern auch mit Rücksicht auf die vom Kunden wahrgenommene Sicherheit gerecht werden – etwa durch visuelle Elemente, die mit Sicherheit assoziiert werden.

	In den nächsten Abschnitten wird nun erläutert werden, welche Funktionen wir uns in Konsequenz überlegt haben und mit welchen Use-Cases diese korrespondieren.

\subsection{Name}
    Der Name der App, „\textbf{K}i\textbf{B}a“, steht – in Anlehnung an eine bekannte Direktbank – ,für eine kundeninteressierte Bank, die mit der Anwendung mehr über ihre Kunden erfahren möchte und mit ihnen in Austausch treten will, um spezifischere Angebote und Beratungen anbieten zu können.
    
\subsection{Plattform}
    Für die Wahl einer geeigneten Plattform – iPad oder iPhone – haben wir uns zunächst überlegt, in welchen Situationen wir die Interaktion mit einer Banking-App für wahrscheinlich hielten. Es erschien uns, als wäre der typische Anwendungsfall für Bankgeschäfte eher stationär, etwa auf dem Sofa oder im Büro, jedenfalls aber in Ruhe.     
    
    Somit überwogen in der Gruppe eindeutig die Argumente für eine iPad-Anwendung. Zudem war es Aufgabe, eine Vision für die Banking-App der Zukunft zu entwickeln. Der Trend geht unserer Meinung nach zum ubiquitären WLAN. Insofern darf davon ausgegangen werden, dass die mobile Konnektivität zukünftig auch beim iPad unproblematisch ist.
    
\subsection{Funktionen}
\subsubsection{Finder}
	Ein intuitiv klarer Anwendungsfall ist der Wunsch eines Nutzers, die nächstliegende Filiale aus der App heraus auf einer Karte zu finden und dorthin navigieren zu können. Um diese Funktionalität noch gegenüber einem bekannten Kartendienst abzuheben, soll schon an dieser Stelle zusätzliche Interaktion mit einer Filiale möglich sein. Über eine Sortenanfrage können Devisen bestellt und aus der Filialseite, welche aus der Karte geöffnet wird, kann ein Termin vereinbart werden.

\subsubsection{Authentifizierung}
	Der Authentifzierungsmechanismus trennt die Funktionen in sicherheitsrelevante und unkritische. Ohne Authentifizierung steht dem Benutzer nur passive Funktionalität zur Verfügung, also etwa der Filialfinder oder die Umsatzanzeige. Um vom vollen Funktionsumfang der App profitieren zu können, muss der Benutzer in eine KiBa-Filiale gehen und von einem Berater einen Sicherheitscode eingeben lassen. Dieser garantiert dann in Kombination mit der App-ID der Anwendung eine eindeutige Zuordnung eines Benutzers an sein Gerät. Ähnlich einer Kreditkarte soll dann bei Verlust des Geräts auch die App selbst jederzeit gesperrt werden können. Insbesondere dient die Aktivierung aber auch dazu, den Kunden in einem Beratungsgespräch besser kennenzulernen und in einer Datenbank einen persönlichen Ansprechpartner festzuhalten.
	
\subsubsection{Self-Service}
	Im Zuge der Plenumsdiskussionen und anschließenden internen Debatten haben wir den zeitlichen Aspekt als zentrale Komponente identifiziert. Lokalität bedeutet, Dinge direkt zur Verfügung gestellt bekommen zu können. Viele Bescheinigungen und Unterlagen im täglichen Leben werden auch heute noch konkret ausgedruckt benötigt. Der typische Ablauf einer Direktbank sieht so aus, eine Anfrage – etwa nach Wertpapierzweitschriften oder Bonität – per Kontaktformular abzuschicken und dann einige Tage auf die entsprechenden Ausdrucke zu warten. Unsere Idee besteht in einer Self-Service-Station innerhalb der Bank, die das Konzept bestehender Automaten erweitert.
	
\begin{figure}
	\centering
	\includegraphics[scale=0.2]{Pictures/SelfService}
	\caption{Piktogramm unserer Self-Service-Station\label{fig:SelfService}}
\end{figure}
	
	Ein Kunde kann sein iPad auf eine Ablage legen und sich mit der Station verbinden, welche eine Art Multifunktionsdrucker beinhaltet, vergleiche Abbildung \ref{fig:SelfService}. Über die App können dann verschiedenste Bescheinigungen ausgedruckt und Konten eröffnet werden. Das entscheidende dabei ist, dass der Kunde durch die Authentifizierung in einer Filiale seinem Gerät zugeordnet wurde. Somit können an derartigen Stationen auch Interaktionen vorgenommen werden, die normalerweise die Identifikation per Lichtbildausweis am Schalter erfordern würden. Somit entsteht hier Mehrwert für alle Stakeholder: Kunden müssen weniger anstehen für standardisierte Abläufe, die Bank spart unter Umständen Personalkosten und kann Mitarbeiter für das Wesentliche, die Beratung, einsetzen.

   Umbuchungen am Sparbuch können üblicherweise nur in einer Filiale vorgenommen werden, überwiesen werden kann nur auf das Sparkonto. Die Greifbarkeit des Sparbuches vermittelt konservativen, besorgten Sparern ein Gefühl von Sicherheit. Gleichzeitig kann es aber durch diese funktionale Einschränkung auch zu unerwünschten Situationen kommen: ist etwa durch eine Fehlkalkulation an einem Samstagabend kein Geld mehr auf dem Girokonto, muss bis zur Öffnung einer Filiale am Montag gewartet werden. Um dem vorzubeugen, soll über die Self-Service-Station auch Geld umgebucht werden können. Da die Station im Vorraum der Filiale steht, ist sie immer zugänglich.   
    
\subsubsection{Individueller Finanzierungsrechner}
	Wie oben erläutert, besteht eine wesentliche Dienstleistung von Filialbanken in der Finanzierung, etwa für Eigenheime. Im Zentrum unserer Überlegungen stand dann auch die Frage, wie eine App dabei helfen kann, diese Dienstleistung für den Endkunden zu verbessern.
	
	Im Ergebnis möchten wir einen individualisierten Kreditrechner anbieten, der den App-Benutzer in die Lage versetzt, mittels individuell berechneter Profildaten für sich selbst Finanzierungsrechnungen durchzuführen. Die Idee dahinter ist, dass ein Kunde zunächst für sich selbst einige Finanzierungsvarianten durchspielen kann. Hat er sich für eine Variante entschieden, kann ein Termin mit dem persönlichen Berater vereinbart werden. Wichtig dabei ist, dass die angebotenen Finanzierungsdaten  mit Rücksicht auf die der Bank zur Verfügung stehenden Informationen so konservativ gewählt sind, dass sie in jedem Falle von der Bank eingehalten werden können.
 
    