\documentclass[a4paper, 12pt]{scrartcl}

%===============================================================================
% Packages
%===============================================================================
\usepackage{fontspec}
\usepackage{geometry}
\usepackage[ngerman]{babel} % Use new german language definition
\usepackage{amsmath, amsfonts, amssymb} % Math packages
\usepackage{graphicx} % For Graphics
\usepackage{xcolor} % For coloring texts
\usepackage{times} % Times New Roman Font
\usepackage{tikz} % The TikZ vector graphic library
\usepackage{fancyhdr} % Fancy headings
\usepackage[absolute]{textpos} % Absolute positioning

%===============================================================================
% Definitions
%===============================================================================
%\newcommand{\semantischerAusdruck}[1]{%
%	... was macht das ... #1

\newcommand{\Stichwort}[1]{%
	\textbf{#1}}
\setlength{\parskip}{6pt}
\renewcommand{\headrulewidth}{0pt}
\linespread{1.5}
\geometry{a4paper}
\geometry{left=25mm,right=25mm,bottom=30mm,top=30mm}
\geometry{footnotesep=10mm}
\geometry{headsep=10mm}
\geometry{head=10mm}
\geometry{bindingoffset=0mm}
\definecolor{kiba}{HTML}{41707C}
\pagestyle{fancy}
\fancyhead[R]{\itshape Projektbericht „KiBa“-App}

%===============================================================================
% Definitions
%===============================================================================
\setmainfont{Times}
\setsansfont{Helvetica Neue Light}
    
%===============================================================================
% Meta Description
%===============================================================================    
\author{%
	Alexander Droste\vphantom{yMö}\\[-12pt]
    Markus Fasselt\vphantom{yöM}\\[-12pt]
    Marco F. Jendryczko\vphantom{yMö}\\[-12pt]
    Konstantin S. M. Möllers \vphantom{yMö}\\[-12pt]
    Michael Schaarschmidt\vphantom{yöM}\\[-12pt]
    Julius Wulk\vphantom{yMö}}
\title{Die „KiBa“-App}
\date{Wintersemester 2013/14}

    
%===============================================================================
% Document
%===============================================================================
\begin{document}
	
    \makeatletter
\thispagestyle{empty}

\begin{center}
	\textcolor{kiba}{\bf\sffamily\fontsize{48pt}{48pt}\selectfont \@title} \\[.5cm]
	\includegraphics[height=3cm]{logo-kiba} \\[.5cm]
	\textcolor{kiba}{\LARGE\sffamily Projektbericht} \\[.5cm]
    {\large\sffamily Angefertigt von} \\[4pt]
    {\Large \@author \\[1.5cm]}
    {\large\sffamily Veranstalter} \\[12pt]
    {\Large Dr. Guido Gryczan} \\[6pt] {\LARGE Dr. Martin Christof Kindsmüller\vphantom{y}} \\[6pt] {\LARGE Christian Zoller} \\[1.5cm]
    {\Large\sffamily \@date}
\end{center}

\vspace{2.5cm}

% UHH-Logo
\begin{textblock*}{85mm}(20mm, 257mm)
	\includegraphics[width=2cm]{logo-uhh}\;{\sffamily Universität Hamburg}
\end{textblock*}

% FBI-Logo
\begin{textblock*}{85mm}(105mm, 257mm)
	\hfill{\sffamily Fachbereich Informatik}\;\includegraphics[width=2cm]{logo-fbi}
\end{textblock*}

\makeatother
    \newpage
    
    \tableofcontents
    \newpage
    
    \section{Aufgabenstellung und Vision}
	Die unser Gruppe übertragene Aufgabe bestand darin, eine iOS-App zu entwickeln, die die Bindung zwischen einer fiktiven Filialbank und ihren Kunden erhöht. In den ersten Ü\-ber\-le\-gung\-en wurde schnell deutlich, dass nahezu jede Dienstleistung einer Filiale auf die ein oder andere Weise von Direktbanken abgebildet werden kann. Das Alleinstellungsmerkmal besteht im direkten menschlichen Kontakt vor Ort und in der persönliche Bindung zu einem möglicherweise vertrauten Berater. 
    
Insofern haben wir uns nach einem Prozess von abwechselnden Gruppendiskussion und externen Feedback dazu entschlossen, diejenigen Features zu betonen, die Berater und Kunden möglichst in der Filiale zusammenbringen, aber auch unabhängig davon genutzt werden können. Auch Direktbanken versuchen inzwischen, anonyme Hotlines zu vermeiden und Kunden einzelnen Beratern zuzuordnen. In Kombination mit niedrigeren Gebühren, kostenlosen Kreditkarten und einfacher Kontoführung ist dies ein schwer zu schlagendes Angebot. Eine App, die Neukunden für eine Filalbank gewinnen soll, muss also einerseits eine ähnlich einfache Kontoverwaltung bieten, außerdem aber noch einen Mehrwert schaffen, der die höheren Gebühren einer Filialbank rechtfertigt.

Ein weiterer, für die Akzeptanz einer App wichtiger Aspekt ist die (vielleicht auch nur vom Kunden gefühlte) Sicherheit. Eine von uns durchgeführte Umfrage unter Studenten technischer Fächer, primär Informatik, ergab, dass die meisten Teilnehmer unabhängig von der Funktionalität schon aus Sicherheitsgründen keine Bankgeschäfte mit einer App erledigen wollten. Die besondere Sensibilität der genannten Zielgruppe für Sicherheitsaspekte darf nicht außer Acht gelassen werden; dennoch ergab sich für uns die Frage, ob hier durch Interaktion in der Filiale nicht zugleich Kundenbindung und Vertrauen in die Sicherheit der Anwendung erhöht werden könnten. 

Insgesamt entstand also die Vorstellung einer Anwendung, die klassische Bankgeschäfte wie Überweisungen beherrscht, gleichzeitig aber versucht, Lokalität und Interaktion mit dem eigenen Berater herzustellen. 
     
    \section{Konzept}
    \subsection{Name}
    Der Name der App, "`\textbf{K}i\textbf{B}a"', steht in Anlehnung an eine bekannte Direktbank für eine kundeninteressierte Bank, die mit der Anwendung mehr über ihre Kunden erfahren möchte und mit ihnen in Austausch treten will, um spezifischere Angebote und Beratungen anbieten zu können.
    
   	\subsection{Gerät}
    Bei der Erstellung eines konkreten Konzepts der Funktionalität haben wir zunächst die Anwendungsfälle besprochen, um die Gerätfrage zu klären. Zweifelsohne gibt es Features, die in einem mobilen Anwendungsfall wahrscheinlicher sind, etwa das Auffinden einer Filiale oder eines Geldautomaten. Eine solche Funktionalität wäre aber nicht filialbankspezifisch. Überhaupt erschien es uns, als wäre der typische Anwendungsfall eher stationär, auf dem Sofa, im Büro, jedenfalls aber in Ruhe. Ein überzeugendes Argument für eine iPad-App ist auch, dass Kunde und Berater in der Filiale zusammen mit der App interagieren und Dinge visualisieren können. Die Vorstellung, ein Kreditangebot auf dem Bildschirm eines iPhones durchzusprechen, ist hingegen eher absurd. 
    
    
    Somit überwogen in der Gruppe ganz eindeutig die Argumente für eine iPad-Anwendung. Zudem war es Aufgabe, eine Vision für die Banking-App der Zukunft zu entwickeln. Der Trend geht unserer Meinung nach zum ubiquitären W-Lan; so gibt es beispielsweise Bestrebungen, öffentliche Netzwerke in der Innenstadt einzurichten. Insofern darf davon ausgegangen werden, dass die mobile Konnektivität zukünftig auch beim iPad zukünftig unproblematisch ist und somit webbasierte Funktionalität auch unterwegs gegeben ist.
    

    \subsection{Funktionen}
    Die Kernfunktionen der App sollen sich auf genau die Bereiche konzentrieren, die einer Direktbank nicht zur Verfügung stehen und somit nicht trivial nachgebaut werden können. Dabei haben wir im Zuge der Plenumsdiskussionen und anschließenden internen Debatten insbesondere den zeitlichen Aspekt als zentrale Komponente identifziert. Viele Bescheinigungen und Unterlagen im täglichen Leben werden auch heute noch konkret ausgedruckt benötigt. Der typische Ablauf einer Direktbank sieht so aus, eine Anfrage (etwa nach Wertpapierzweitschriften) per Kontaktformular abzuschicken und dann einige Tage auf die entsprechenden Ausdrucke zu warten. Unsere Idee besteht in einer Self-Service Station innerhalb der Bank, die das Konzept bestehender Automaten erweitert. Ein Kunde kann sein Ipad auf eine Ablage legen und sich mit der Station verbinden. Die Station beinhaltet einen Multifunktionsdrucker und per App können verschiedenste Bescheinigungen ausgedruckt werden. 
    
    Ein wesentliches Produkt von Filialbanken ist das klassische Sparbuch. Umbuchungen können üblicherweise nur in einer Filiale vorgenommen werden, überwiesen werden kann nur auf das Sparkonto. Die Greifbarkeit des Sparbuches vermittelt konservativen, besorgten Sparern ein Gefühl von Sicherheit. Gleichzeitig kann es aber durch diese funktionale Einschränkung auch zu unerwünschten Situationen kommen: ist etwa durch eine Fehlkalkulation an einem Samstagabend kein Geld mehr auf dem Girokonto, muss bis zur Öffnung einer Filiale am Montag gewartet werden, um Guthaben umbuchen zu können. Um dem Vorzubeugen, soll über die Self-Service Station auch Geld umgebucht werden können. Da die Station im Vorraum der Filiale steht, ist sie ganztägig zugänglich. Auf diese Weise bleibt einerseits das Sparbuch als vertrauenswürdige Marke erhalten, die in der Wahrnehmung misstrauischer Benutzer von den Verwerfungen der Online-Kriminalität unberührt bleibt, andererseits ist die Verfügbarkeit erheblich verbessert.
    
    Ebenfalls auf die Bereitstellung von Service-Dienstleistungen zielt der interaktive Filialfinder ab. Aus der Karte heraus sollen Anfragen an eine bestimmte Filiale in der Umgebung er möglicht werden, indem etwa ein Beratungstermin reserviert wird und dann ohne Anstehen erfolgen kann. Eine Funktionalität dieser Art ist 
      
    Als Startbildschirm für die App ist ein Dashboard vorgesehen, das einen graphischen Überblick über Vermögensverlauf und Transaktionen bereitstellt. Hilfreich war hier die Überlegung, dass die meisten Benutzer ihren Kontostand grob kennen und weniger an einer Zahl als vielmehr an den Entwicklungen interessiert sind. Ist der Benutzer nicht eingeloggt, erscheint an dieser Stelle ein Mockup mit der Aufforderung, sich einzuloggen.
    
    Das Kerngeschäft von Filialbanken besteht in der Finanzierung, etwa für Eigenheime. Im Zentrum unserer Überlegungen stand dann auch die Frage, wie eine App dabei helfen kann, diese Dienstleistung für den Endkunden zu verbessern. Im Ergebnis möhten wir einen individualisierten Kreditrechner anbieten, der den App-Benutzer in die Lage versetzt, mittels individuell berechneter Profildaten für sich selbst Finanzierungsrechnungen durchzuführen. Die Idee dahinter ist, dass ein Kunde zunächst für sich selbst einige Finanzierungsvarianten durchspielen kann. Hat er sich eine Variante überlegt, kann ein Termin mit dem persönlichen Berater vereinbart werden und dabei optional gleich der Finanzierungsvorschlag exportiert werden. Im Filialgespräch kann der Berater dann noch individuelle Ratschläge bezüglich Laufzeit und Umfang einer Finanzierung geben.
    
	Eng zusammenhängend mit dem Finanzierungsprofil steht die Aktivierung ("`Authentifikation"') der App in einer Filiale. Ohne eine solche Aktivierung steht dem Benutzer nur passive Funktionalität zur Verfügung, also etwa der Filialfinder und die Umsatzanzeige. Um die App voll nutzen zu können, muss der Benutzer in eine KiBa-Filiale und von einem Berater einen Sicherheitscode eingeben lassen. Ähnlich einer Kreditkarte soll dann bei Gerätverlust auch die App selbst jederzeit gesperrt werden können. Insbesondere dient die Aktivierung aber auch dazu, den Kunden in einem Beratungsgespräch besser kennenzulernen und in einer bankseitigen Datenbank einen persönlichen Ansprechpartner festzuhalten. Aus der App kann dann direkt ein Termin mit dem eingetragenen Berater vereinbart werden. Auch ein Nachrichten-System ist denkbar, um einzelne Fragen direkt zu klären.
    
    
    \subsection{Mockups}
    
    
    \section{Projektplan}

    \section{Umsetzung}
    \subsection{Setup}
    
    \subsection{Scrum}
    Im Rahmen des Projekts wurden wir mit der Srum-Methodik vertraut gemacht und hatten Gelegenheit,eine einführende Zertifizierung zu absolvieren. Während der erste Hälfte des Semesters standen Vision, Konzeption und Lernprozesse im Vordergrund. Mit Beginn der eigentlichen Implementierung in der zweiten Semesterhälfte bestand dann die Herausforderung darin, den Scrum-Prozess auf die zeitlichen Gegegebenheiten einer Gruppe von Studenten mit unterschiedlichen Stundenplänen abzustimmen. Zunächst einmal war es, abgesehen von Ausnahmen, kaum möglich, sich außerhalb der festen Projekttermine in voller Gruppenstärke zu einem festen Termin zu treffen. 
    
    Um in unseren Arbeitsabläufen dennoch eine gewisse Kontinuität herzustellen, haben wir zwei wöchentliche Termine festgelegt, bei denen immer mindestens die Hälfte der Gruppe anwesend sein konnte. Als Sprintdauer haben wir zwei Wochen festgelegt, hauptsächlich basierend auf der Erfahrung, wie lange einzelne Features in anderen universitären Projekten gedauert haben. In gewisser Hinsicht haben wir hier also eine Scrum-ähnliche Retrospektive benutzt, um unseren ersten Sprint zu planen.
    
    Das Scrum-Framework verbietet eine Aufteilung in Unterteams. Zugleich wurde uns aber vermittelt, die Wegnahme einzelner Scrum-Elemente oder Missachtung einzelner Regeln bedeute, gar nicht mehr Scrum zu benutzen, da Scrum unteilbar sei. Ein weiteres Problem ergab sich dadurch, dass Scrum unserem Verständnis nach vorsieht, dass das Entwicklungsteam zu Beginn der Arbeit bereits alle notwendigen technischen Kompetenzen zur Umsetzung eines Projekts besitzt. Zuverlässige Schätzungen für die Entwicklungszeit sind andernfalls schwer möglich. Universitäre Projekte haben aber natürlich auch immer eine Komponente, in der sich die Teilnehmer selbstständig das Wissen erarbeiten, das zur Vollendung einer Aufgabe notwendig ist. Diese Überlegung haben wir in unsere Schätzung mit einbezogen. Dennoch ist es an technisch schwierigen Stellen schwer abzusehen, wie lange es dauern wird, eine spezielle Lösung zu finden. Insofern sind Hilfen wie Burdown-Charts dann nicht besonders indikativ dafür, wie sich bisher Erledigtes zu verbleibenden Aufgaben mit Blick auf die Einarbeitungsschwierigkeiten verhält.
    \subsection{Architektur}
    
    \subsection{Bedienung}
    \section{Fazit}
    \subsection{Produkt}
    \subsection{Projekt}
    
\end{document}